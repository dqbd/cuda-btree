\section{B-Link-Tree}



Previous approaches include locking a subtree of highest affected node \cite{samadi1976b}, which, albeit straightforward, severely reduced concurrency.

In order to alleviate the bottleneck without risking inconsistency, \textit{B-Link-Tree} relaxes the definition of B$^+$Trees. As explained by Graefe \cite{goetz-tech}:

\begin{definition}
  B-Link-Tree is a B$^+$Tree with following properties:
  \begin{enumerate}
    \item Each node $x$ has additional attributes:
          \begin{itemize}
            \item $x.sibling$, a pointer to a right sibling node at the same depth,
            \item $x.highkey$, the upper bound of all keys found in the subtree rooted by $x$ (every key found in $x$ is less then $x.highkey$).
          \end{itemize}
    \item Does not require locks nor latches for reading.
  \end{enumerate}
\end{definition}

\begin{figure}[H]
  \centering
  \begin{tikzpicture}[
      bnode/.style = {
          draw,
          text width=1em,
          align=center,
          rectangle split,
          rectangle split horizontal,
          rectangle split parts = 4,
        },
    ]

    \node[bnode](root) {
      \nodepart{one} $65$
      \nodepart{two} $72$
      \nodepart{three} $77$
      \nodepart{four}
      \nodepart{five}
    };

    \draw[->](root.south west) -- +(-3.6,-1)
    node[bnode, anchor=north](a){
        \nodepart{one} $51$
        \nodepart{two} $52$
        \nodepart{three}
        \nodepart{four}
      };

    \draw[->](root.one split south) -- +(-1.2,-1)
    node[bnode, anchor=north](b) {
        \nodepart{one} $65$
        \nodepart{two} $68$
        \nodepart{three}
        \nodepart{four}
      };

    \draw[->](a.east) -- (b.west);

    \draw[->](root.two split south) -- +(1.2,-1)
    node[bnode, anchor=north](c) {
        \nodepart{one} $72$
        \nodepart{two} $73$
        \nodepart{three}
        \nodepart{four}
      };

    \draw[->](b.east) -- (c.west);

    \draw[->](root.three split south) -- +(3.6,-1)
    node[bnode, anchor=north](d) {
        \nodepart{one} $77$
        \nodepart{two} $78$
        \nodepart{three} $79$
        \nodepart{four}
      };

    \draw[->](c.east) -- (d.west);

  \end{tikzpicture}
  \caption{B-Link-Tree with $\mathit{Order} = 5$}
\end{figure}


Splitting during node insertion is divided into two independent steps: splitting a node and inserting the split node with its new separator key to the parent node. When splitting node $x$, a new right sibling node $y$ is created. The node $y$ inherits the high key from the split node $x$, whereas the first key of $y.keys$ is used as a new $x.highkey$. Thus, an internal node does exist without a parent in between the operations. $x.highkey$ is used to be able to traverse the newly split node and its subtree, even when the split operation is not complete.

\todo{Add insertion pseudocode}

As the final step of node splitting, the separator key and the newly split node $y$ are inserted in the parent node. 

\todo{Add search pseudocode}

Tree traversal is modified to honor $x.highkey$ by returning the node at $x.sibling$ when the target key $k$ is larger or equal than $x.highkey$.
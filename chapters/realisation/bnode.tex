\section{Data Structure and Memory Allocation}

This section describes the proposed data structure, found in the \cref{lst:bnode}. Comments and helpers methods were removed for the sake of brevity.

\begin{listing}
  \begin{minted}{cpp}
template <typename KeyType, typename ValueType, size_t Order>
struct BNode {
  BNode * mSibling;
  KeyType mHighKey;

  uint16_t mHighKeyFlag;
  uint16_t mLeaf;
  uint16_t mSize;
  uint16_t mWriteLock;

  KeyType mKeys[Order];
  ValueType mValues[Order];
  BNode * mChildren[Order]; 
}
    \end{minted}
  \caption{The \code{BNode} struct}\label{lst:bnode}
\end{listing}

\codecpp{uint16_t} have been chosen in favor of smaller data types, as most instructions in the \acrshort{isa} do not support operand types smaller than 16-bits and instead convert them to larger data types via a \mintinline{asm}{cvt} statement.

In the B-Link-Tree variant, \codecpp{mSibling} and \codecpp{mChildren} are using the \codecpp{volatile} qualifier to avoid incorrect memory access optimization by the compiler and ensure the correctness of the tree during operations. This qualifier does incur a performance hit, however.
